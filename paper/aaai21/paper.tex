\def\year{2021}\relax
%File: formatting-instruction.tex
\documentclass[letterpaper]{article} % DO NOT CHANGE THIS
\usepackage{aaai20}  % DO NOT CHANGE THIS
\usepackage{times}  % DO NOT CHANGE THIS
\usepackage{helvet} % DO NOT CHANGE THIS
\usepackage{courier}  % DO NOT CHANGE THIS
\usepackage[hyphens]{url}  % DO NOT CHANGE THIS
\usepackage{graphicx} % DO NOT CHANGE THIS
\urlstyle{rm} % DO NOT CHANGE THIS
\def\UrlFont{\rm}  % DO NOT CHANGE THIS
\usepackage{graphicx}  % DO NOT CHANGE THIS
\frenchspacing  % DO NOT CHANGE THIS
\setlength{\pdfpagewidth}{8.5in}  % DO NOT CHANGE THIS
\setlength{\pdfpageheight}{11in}  % DO NOT CHANGE THIS
%\nocopyright
%PDF Info Is REQUIRED.
% For /Author, add all authors within the parentheses, separated by commas. No accents or commands.
% For /Title, add Title in Mixed Case. No accents or commands. Retain the parentheses.
 \pdfinfo{
/Title (Pragmatic Code Autocomplete)
/Author (Anonymous)
} %Leave this	
% /Title ()
% Put your actual complete title (no codes, scripts, shortcuts, or LaTeX commands) within the parentheses in mixed case
% Leave the space between \Title and the beginning parenthesis alone
% /Author ()
% Put your actual complete list of authors (no codes, scripts, shortcuts, or LaTeX commands) within the parentheses in mixed case. 
% Each author should be only by a comma. If the name contains accents, remove them. If there are any LaTeX commands, 
% remove them. 

% DISALLOWED PACKAGES
% \usepackage{authblk} -- This package is specifically forbidden
% \usepackage{balance} -- This package is specifically forbidden
% \usepackage{caption} -- This package is specifically forbidden
% \usepackage{color (if used in text)
% \usepackage{CJK} -- This package is specifically forbidden
% \usepackage{float} -- This package is specifically forbidden
% \usepackage{flushend} -- This package is specifically forbidden
% \usepackage{fontenc} -- This package is specifically forbidden
% \usepackage{fullpage} -- This package is specifically forbidden
% \usepackage{geometry} -- This package is specifically forbidden
% \usepackage{grffile} -- This package is specifically forbidden
% \usepackage{hyperref} -- This package is specifically forbidden
% \usepackage{navigator} -- This package is specifically forbidden
% (or any other package that embeds links such as navigator or hyperref)
% \indentfirst} -- This package is specifically forbidden
% \layout} -- This package is specifically forbidden
% \multicol} -- This package is specifically forbidden
% \nameref} -- This package is specifically forbidden
% \natbib} -- This package is specifically forbidden -- use the following workaround:
% \usepackage{savetrees} -- This package is specifically forbidden
% \usepackage{setspace} -- This package is specifically forbidden
% \usepackage{stfloats} -- This package is specifically forbidden
% \usepackage{tabu} -- This package is specifically forbidden
% \usepackage{titlesec} -- This package is specifically forbidden
% \usepackage{tocbibind} -- This package is specifically forbidden
% \usepackage{ulem} -- This package is specifically forbidden
% \usepackage{wrapfig} -- This package is specifically forbidden
% DISALLOWED COMMANDS
% \nocopyright -- Your paper will not be published if you use this command
% \addtolength -- This command may not be used
% \balance -- This command may not be used
% \baselinestretch -- Your paper will not be published if you use this command
% \clearpage -- No page breaks of any kind may be used for the final version of your paper
% \columnsep -- This command may not be used
% \newpage -- No page breaks of any kind may be used for the final version of your paper
% \pagebreak -- No page breaks of any kind may be used for the final version of your paperr
% \pagestyle -- This command may not be used
% \tiny -- This is not an acceptable font size.
% \vspace{- -- No negative value may be used in proximity of a caption, figure, table, section, subsection, subsubsection, or reference
% \vskip{- -- No negative value may be used to alter spacing above or below a caption, figure, table, section, subsection, subsubsection, or reference

\setcounter{secnumdepth}{0} %May be changed to 1 or 2 if section numbers are desired.

% The file aaai20.sty is the style file for AAAI Press 
% proceedings, working notes, and technical reports.
%
\setlength\titlebox{2.5in} % If your paper contains an overfull \vbox too high warning at the beginning of the document, use this
% command to correct it. You may not alter the value below 2.5 in
\title{Pragmatic Code Autocomplete}
%Your title must be in mixed case, not sentence case. 
% That means all verbs (including short verbs like be, is, using,and go), 
% nouns, adverbs, adjectives should be capitalized, including both words in hyphenated terms, while
% articles, conjunctions, and prepositions are lower case unless they
% directly follow a colon or long dash
\author{
  Blind submission
}
 \begin{document}

\maketitle

\begin{abstract}
  One of the defining features of programming languages is that their syntax
  is captured by unambiguous grammars. In contrast, natural languages are
  highly ambiguous. Nevertheless, humans reliably utilize utterances with multiple
  possible interpretations, since speakers and listeners reason in context.
  In this work, we aim
  at making programming languages more concise by allowing programmers to use
  ambiguous one-character abbreviations for common keywords and identifiers.
  First, the system proposes a set of strings that can be abbreviated by the user.
  Using only 100 abbreviations, we observe that a corpus of Python code can
  be compressed by 13\%.
  We then use a contextual sequence to sequence model to rank potential expansions of
  abbreviations. In an offline task of first abbreviating a line of code and then
  predicting the original line, our model achieves 98\% top-1 accuracy.
  We evaluate the usability of our system in a user study, integrating it in
  Microsoft VSCode, a popular code text editor.
  We measure users' typing time using either our system, VSCode's default autocomplete
  feature, and none of these features. Our system improved typing time
  by 10\% compared to VSCode's default autocomplete, and by 7\% compared to not
  using an autocomplete feature.
\end{abstract}

\section{Introduction}

Natural languages evolved to be efficient over time.
One key aspect is that they are highly ambiguous, but listeners use
pragmatic reasoning to attribute meaning to utterances.
Therefore, even extremely short utterances can carry considerable meaning.

In contrast, programming languages have not gone through such a process.
In particular, each program has a single interpretation.
This makes it easier on compilers, but forces programs to be more verbose,
thus programmers have to type more.

To mitigate this issue, development environments integrate autocomplete systems.
However, the HCI literature points out that such systems do not necessarily make
typing faster. In fact, we (hopefully) confirm this finding in our user
study: using VSCode's default autocomplete made users 10\% slower compared
to not using any autocomplete feature at all.

In this work, we take a different approach for making programmers able to type
code more efficiently. First, we observe that keywords and identifiers
constitute 75\% of the characters in Python code in a sample of 1000 repositories
from Github. Furthermore, the distribution of uses of such keywords and
identifiers is highly skewed. Using an algorithm we describe in
Section~\ref{sec:approach}, we find a set of $100$ keywords such that
75\% of all lines of Python code from our corpus contain at least one of them.
Moreover, if these keywords are abbreviated to just their first character,
the entire corpus is compressed by 13\%.
These abbreviations would introduce ambiguity: for instance, both \texttt{return}
and \texttt{range} would be abbreviated as \texttt{r}).
However, by using the context in which the abbreviations are used,
we are able to obtain a model that finds the correct expansion of an abbreviated
line of code with 96\% top-1 accuracy.

Contributions:

\begin{itemize}
  \item New formulation of the autocomplete problem for code
    (instead of language model-like, we make it abbreviations-based).
  \item A model that can propose and expand abbreviations with very high
    accuracy.
  \item An evaluation of that model both offline and with users, showing it
    improves typing time compared to traditional autocomplete and no autocomplete.
\end{itemize}

\section{Related Work}
\label{sec:rw}

Natural language: efficiency improvement over time through
the shortening of common words, ambiguity and pragmatics.

Autocomplete for natural language: HCI papers that show it's not always useful,
Mina's paper with keyword-based autocomplete formulation.

Autocomplete for code: offline tasks, no user studies.

\section{Approach}
\label{sec:approach}

Formulation: 

\begin{enumerate}
  \item{Read corpus and propose abbreviatable set $S$}
  \item{Task: given dataset $S$, abbreviate lines of code and learn to expand}
  \item{Context $c$: previous $K$ lines of code}
  \item{Conditional language model: directly model $p(l | s, c)$ with seq2seq + attention + context (ConcatCell)}
  \item{Use model + beam search to expand abbreviations}
\end{enumerate}

\section{Experiments}
\label{sec:exp}

Dataset: Python repositories from Github.

Research questions:

\begin{enumerate}
  \item{Can the model learn to expand abbreviated lines with high accuracy?}
  \item{Is context important?}
  \item{Is it usable and useful?}
\end{enumerate}

RQ1: offline task: show that model can expand abbreviated lines from the dataset
with ~98\% accuracy.

RQ2: context ablation. Vary $K$.

RQ3: user study

\section{Conclusion}
\label{sec:conclusion}

\end{document}
